\documentclass[12pt,a4paper,notitlepage]{article}
\usepackage[latin1]{inputenc}
\usepackage[english]{babel}
\usepackage{graphicx}
\usepackage{fullpage}
\usepackage[final]{pdfpages}
\usepackage[font=sl]{caption}
\usepackage{listings}
\usepackage{hyperref}
\hypersetup{
    colorlinks,%
    citecolor=black,%
    filecolor=black,%
    linkcolor=black,%
    urlcolor=black
}

\setlength{\parindent}{0pt}
\setlength{\parskip}{1.8ex plus 0.5ex minus 0.2ex}

\title{\huge{Dominion}\\
\large{Analysis, Design and Software Architecture}
}
\author{
Jakob Melnyk, jmel@itu.dk\\
Christian Jensen, chrj@itu.dk\\
Frederik Lysgaard, frly@itu.dk
}
\date{November 25th, 2011}
\begin{document}
\maketitle
\vfill
\section*{Abstract}
This project is about a virtual representation of the card game Dominion in C\#. Dominion is a turn-based, deck-building game, where the objective is to gather more points than the other players. The game is played by 2 - 4 players.
\clearpage
\tableofcontents
\pagebreak
\section{Requirements}
\subsection{Mandatory}
Must be able to play a full game of Dominion
\begin{itemize}
\item The must support 2 players in Hot-Seat configuration
\item At least 10 Kingdom cards must work
\item The game must be playable in a Picture-based GUI
\end{itemize}
\subsection{Secondary}
High priority
\begin{itemize}
\item Be able to play the game with 3 or more players
\item Be able to use at least 20 Kingdom cards
\item Be able to select Game Mode
\begin{itemize}
\item Be able to play 'First Game' Card-set
\item Be able to play with 10 randomly select Kingdom cards
\end{itemize}
\item Be able to see all Available Kingdom cards without scrolling
\end{itemize}
Medium priority
\begin{itemize}
\item Be able to view a Tooltip when mousing over any Card in the game
\item Be able to play the game over LAN
\item Be able to use all Kingdom cards (from the original version of the game)
\item Be able to play all the Card-sets defined in the original rules
\end{itemize}
Low priority
\begin{itemize}
\item Be able to Draft Kingdom cards
\item Be able to play the game over the Internet
\item Be able to select different screensizes
\item Be able to play in fullscreen
\item Be able to create a User, that is saved across multiple games, with the following information:
\begin{itemize}
\item Statistics
\item Options (if any)
\item Achievements (if implemented)
\end{itemize}
\item Be able to support Extensions of the basic game
\item Implement Achievements for funny and/or hard accomplishments
\end{itemize}
\section{Overview}
This project is about our virtual representation of the card game Dominion. Dominion is a turn-based, deck-building game. The objective of the game is to use Action cards to improve your chances or damage the opponent players and using Treasure cards to buy more powerful Action/Treasure/Victory cards to gain the upper hand. 

Frederik Lysgaard is the guy responsible for the design of our graphical interface. He is also the best Dominion player in our group. Because of this, he knows a lot of the usual strategies and is our general "go-to" guy when it comes to the tactics of the game.

Christian Jensen is responsible for implementing the way the different cards interact with the state of the game when used. Christian is also the guy who will be looking into the networking portion of the project if/when it becomes relevant.

Jakob Melnyk is responsible for modeling the state of the game and the communication between the GUI and the model (in our model-view-controller architecture). Jakob  Melnyk is also the "version-control-guy", the person with the final word in discussions and the general log-keeper for the group.
\section{Dictionary}
\subsection{General terms}
This section describes the general "out-of-game" terms.
\begin{description}
\item[Achievements] An achievement is token rewarded for funny and/or hard accomplishments within the game.
\item[Card-set] A card-set is 10 different Kingdom cards. Card-sets are used to create a different play experience every time you play.
\item[Dominion] The card-game we are making a virtual representation of. A link to the full rules can be found at Rio Grande Games \cite{dominionRules}.
\item[Draft] Drafting is done by player 1 selecting one Kingdom card to be used in the game, then player 2 selects a Kingdom card, player 3 selects a Kingdom card, player 4 selects a Kingdom card, then back to player 1. This cycle repeats until a set number of Kingdom cards have been selected.
\item[Extensions] Expansion packs add additional types of cards to the pool of cards.
\item[Game Mode] There are different possible game modes: draft, random card selection and predefined card-sets. These are selected before the game starts.
\item[Hot-Seat] Hot-Seat is the act of having 2 or more players play on the same computer. The active player "sits" in the hot-seat while playing, then passing the spot to the next player when his turn ends.
\item[Picture-based GUI] A pictured-based GUI is a visual representation of the state of the game. The different cards are shown as pictures in the GUI.
\item[Statistics] Statistics such as number of games played, numbers of games won/lost, and other similar data about gameplay.
\item[Tooltip] A box with text describing something in the GUI in detail.
\item[User] A user is an entity storing statistics and achievements over the course of different games.
\end{description}
\subsection{In-Game terms}
This section describes the types of cards, supply and other "in-game" terms.
\begin{description}
\item[Available]
\item[Card]
\begin{description}
\item[Curse Card]
\item[Kingdom Card]
\begin{description}
\item[Action Card]
\item[Action-Reaction Card]
\item[Kingdom Victory Card]
\end{description}
\item[Treasure Card]
\item[Victory Card]
\end{description}
\item[Supply]
\item[Trash]
\end{description}
%\section{Example}
%\section{Revision History}
\begin{thebibliography}{9}
\bibitem{dominionRio} http://www.riograndegames.com/games.html?id=278
\bibitem{dominionRules} http://www.riograndegames.com/uploads/Game/Game\_278\_gameRules.pdf
\end{thebibliography}
\end{document}